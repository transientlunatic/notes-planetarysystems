
\section{Vertical motion against gravity}
\label{sec:vert-moti-against}

Consider a rocket with thrust velocity perpendicular to the
gravitational field. The force equation  is
\begin{equation}
  \label{eq:1}
  \dv{v}{t}= \frac{F-Mg}{M}
\end{equation}
For $Mg$ the instantaneous weight of the rocket. Hence
\begin{equation*}
  \label{eq:2}
  \dv{v}{t} = -v~E \frac{\dot{M}}{M}-g
\end{equation*}
then
\begin{equation*}
  \dd{v} = - v~E \frac{\dd{M}}{M} - g \dd{t}
\end{equation*}
and integrating both parts from $t=0$ to $t$,
\begin{equation*}
  v(t) = - v~E \int_{M_0}^M \frac{\dd{M'}}{M'} - \int_0^t g \dd{t'}
\end{equation*}
Assuming that $g$ is constant over the flight,
\begin{align*}
  v(t) &= v~E \log( \frac{M_0}{M(t)} )-gt \\
&= v~E \log(\frac{M_0}{M(t)}) - g \frac{M_0}{\dot{M}} \qty( 1 - \frac{M}{M_0})
\end{align*}
with the second part of the equation the gravity loss term.

\section{Thrust-to-weight ratio}
\label{sec:ttw-ratio}

In order to optimise the amount of energy lost due to gravity we can
consider the \emph{thrust-to-weight ratio}, 
\begin{equation}
  \label{eq:3}
  \psi = \frac{F}{g M_0} = \frac{v~E \dot{M}}{g M_0}
\end{equation}
which allows us to write
\begin{equation}
  \label{eq:4}
  \frac{v(t)}{v~E} = \log( \frac{M_0}{M(t)} ) - \frac{1}{\psi} \qty( 1 - \frac{M}{M_0} )
\end{equation}

\section{Vertical range}
\label{sec:vertical-range}

To calculate the vertical range of a rocket we proceed as before by
integrating the velocity over $t$
\begin{align*} 
s &= \int v(t) \dd{t} = v~E \int_0^t \log( \frac{M_0}{M(t)} ) \dd{t} - \int_0^t g t \dd{t'} \\
&=v~E \frac{M_0}{\dot{M}} \qty[ 1 - \frac{M}{M_0} \qty( \log( \frac{M_0}{M(t)}) + 1 ) ] - \half g t^2 \\
&=v~E \frac{M_0}{\dot{M}} \qty[ 1 - \frac{M}{M_0} \qty( \log( \frac{M_0}{M(t)}) + 1 ) ] \\ & \quad- \frac{g}{2} \frac{M_0^2}{\dot{M}^2} \qty( 1 - \frac{M}{M_0})^2
\end{align*} assuming $g$ is constant.

\section{Launch aspects}
\label{sec:launch-aspects}

Consider the kinetic energy of the payload,
\begin{align*}
  \half M \Delta v^2 &= \half M v~E^2 \log^2\qty(\frac{M_0}{M}) \\ &= \half M v~E^2 \log^2\qty(R)
\end{align*}
For $R$ the rocket parameter. Normalising this by the rest-frame
energy of the propellant in the rocket,
\[ \half (M_0-M) v~E^2 \]
we find the ratio
\begin{equation}
\label{eq:5} \phi(R) = \frac{M}{M_0-M} \log^2(R) = \frac{\log^2(R)}{R-1} 
\end{equation}

\begin{figure}[b]
  \centering
  \begin{tikzpicture}
    \begin{axis}[xmin=0, ymin=0, height = 5cm, width=\columnwidth, xlabel=$R$, ylabel=$\phi(R)$]
      \addplot[mark=none, draw=muted-green, ultra thick, domain=1:10] {(ln(x)*ln(x))/(x-1)};
    \end{axis}
  \end{tikzpicture}
  \caption{$\phi(R)$ as a function of the rocket parameter, $R$.}
  \label{fig:rocket-ratio}
\end{figure}

By differentiating we find the optimal $R$ to be
\begin{equation}
  \label{eq:6}
  \dv{\phi}{R} = 0 = - \frac{\log^2(R)}{(R-1)^2} + \frac{2 \log(R)}{R(R-1)}
\end{equation}
So the optimal $R = R~C$ satisfies
\begin{equation}
  \label{eq:7}
  \log{R~C} = 2 - \frac{2}{R~C}
\end{equation}
so $R~C \approx 4.9$.

It is also important that the payload obtains a horizontal speed in
order to remain in orbit. This requires an inclined launch, with two
components of motion,
\[
\dv{v_z}{t} = \frac{F \sin(\theta)-Mg}{M}, \qquad \dv{v_x}{t} = \frac{F \cos(\theta)}{M} 
\]
for $\theta$ the pitch angle from $\theta=0$ horizontal. Thus
\begin{align*}
v_z(t) &= v~E \sin(\theta) \log( \frac{M~0}{M~E} ) - gt \\ v_x(t) &= v~E \cos(\theta) \log(\frac{M_0}{M})
\end{align*}
assuming $v(0)=0$. The total speed of the rocket is then
\begin{align*}
 v(t) &= \sqrt{v^2_x + v^2_z} \\ &= \qty( v~E^2 \qty( \log(\frac{M_0}{M}))^2 - 2 v~E g t \sin(\theta) \log( \frac{M_0}{M}) + g^2 t^2 )^{\half} 
\end{align*}

The path angle of the flight is related to the ratio of the vertical
to the horizontal velocity, so
\begin{align*} \tan(\gamma) &= \frac{v~E \sin(\theta) \log(\frac{M_0}{M})-gt}{v~E \cos(\theta) \log(\frac{M_0}{M})} \\
&= \tan(\theta) - \frac{gt}{v~E \cos(\theta) \log( \frac{M_0}{M})} \\
& \approx \tan(\theta) - \frac{g M_0}{v~E \dot{M} \cos(\theta)} \quad \text{early in flight}
\end{align*}

We can define the angle of attack as the angle between the thrust and
the flight path, which tends to zero as the burn progresses.

The drag, $F~D$, and lift, $F~L$, are
\[ F~D = C~D A \frac{\rho v^2}{2}, \qquad F~L = C~L A \frac{\rho v^2}{2} \]
where 
\[ \frac{\rho v^2}{2} \] is the dynamic pressure, $C~D$ the drag
coefficient, $C~L$ the lift coefficient, and $A$ the cross-sectional
area.


\section{The gravity turn}
\label{sec:gravity-turn}

A gravity turn is used to rotate the angle of a trajectory using the
gravitational effect of the Earth. Consider the equations from earlier for a constant pitch angle, but substitute the flight path angle, $\gamma$ for the pitch angle, $\theta$, so
\begin{align*}
  \dv{v_z}{t} &= \frac{F}{M} \sin(\gamma) - g = \frac{F}{M} \frac{v_z}{v} -g \\
\dv{v_x}{t} &= \frac{F}{M} \cos(\gamma) =\frac{F}{M} \frac{v_x}{v}
\end{align*}
then multiplying each by their respective velocities, and summing them,
\[ v_z \dv{v_z}{t} + v_x \dv{v_x}{t} = v \dv{v}{t} = \frac{F v}{M} - gv_z \]
Then, the expression for $v$ becomes
\[ \dv{v}{t} = \frac{F}{M} - g \frac{v_z}{v} = \frac{F}{M} - g \sin(\gamma) \]
Multiplying $\dv{v_x}{t}$ by $v_z$ and vice versa, then subtracting one from the other,
\[ v_x \dv{v_z}{t} - v_z \dv{v_x}{t} = - g v_x \]
Since $\tan(\gamma) = v_z / v_x$,
\[ \dv{t} \tan(\gamma) = \frac{\dot{v}_z}{v_x} - \frac{v_z \dot{v}_x}{v_x^2} = \frac{v_x \dot{v}_z - v_z \dot{v}_x}{v_x^2} \]
and the left hand side can be rewritten
\[ \dot{\gamma} (1-\tan[2](\gamma) ) = \dot{\gamma} \qty( 1+ \frac{v_z^2}{v_x^2} ) = \dot{\gamma} \frac{v^2}{v^2_x} \]
Thus, combined,
\[ \dv{\gamma}{t} = - \frac{g v_x}{v^2} = - \frac{g}{v} \cos(\gamma) \]

The equations governing the flight path are non-linear, so some
simplifications are required to solve them.

\section{A uniformly changing path angle}
\label{sec:unif-chang-path}

Suppose $\dot{\gamma} = -C$ is constant, so $\gamma$ changes uniformly
with time. Then
\[ C = \frac{g}{v} \cos(\gamma) \]
so
\[ \dot{v} = - \frac{g}{C} \sin(\gamma) \dot{\gamma} = g \sin(\gamma) \]
Thus
\[ \dot{v} = \frac{F}{M} - \dot{v} \implies 2 \dot{v} = \frac{F}{M} \]
So
\[ \dot{v} = \frac{F}{2M} = \half v~E \frac{\dot{M}}{M} \]
which has a solution
\[ v(t) = v_0 + \half v~E \log(\frac{M_0}{M}) \]
Which, given $\cos(\gamma) = Cv/g$,
\begin{equation}
  \label{eq:58}
  \cos(\gamma(t)) = \cos(\gamma_0) + C \frac{v~E}{2 g} \log(\frac{M_0}{M})
\end{equation}

\section{Orbital injection}
\label{sec:orbital-injection}

Once the atmosphere is sufficiently tenuous orbital injection can
occur, with the perigee of the orbit an important deciding parameter.
For a circular orbit,
\[ v = \sqrt{\frac{GM}{R}+h} \]
and for a parabolic orbit
\[ v = \sqrt{\frac{2GM}{R} + h} \] with everything in between being an
elliptical orbit.

%%% Local Variables: 
%%% mode: latex
%%% TeX-master: "../project"
%%% End: 
