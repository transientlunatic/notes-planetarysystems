
\section{Vertical motion against gravity}
\label{sec:vert-moti-against}

Consider a rocket with thrust velocity perpendicular to the
gravitational field. The force equation  is
\begin{equation}
  \label{eq:1}
  \dv{v}{t}= \frac{F-Mg}{M}
\end{equation}
For $Mg$ the instantaneous weight of the rocket. Hence
\begin{equation*}
  \label{eq:2}
  \dv{v}{t} = -v~E \frac{\dot{M}}{M}-g
\end{equation*}
then
\begin{equation*}
  \dd{v} = - v~E \frac{\dd{M}}{M} - g \dd{t}
\end{equation*}
and integrating both parts from $t=0$ to $t$,
\begin{equation*}
  v(t) = - v~E \int_{M_0}^M \frac{\dd{M'}}{M'} - \int_0^t g \dd{t'}
\end{equation*}
Assuming that $g$ is constant over the flight,
\begin{align*}
  v(t) &= v~E \log( \frac{M_0}{M(t)} )-gt \\
&= v~E \log(\frac{M_0}{M(t)}) - g \frac{M_0}{\dot{M}} \qty( 1 - \frac{M}{M_0})
\end{align*}
with the second part of the equation the gravity loss term.

\section{Thrust-to-weight ratio}
\label{sec:ttw-ratio}

In order to optimise the amount of energy lost due to gravity we can
consider the \emph{thrust-to-weight ratio}, 
\begin{equation}
  \label{eq:3}
  \psi = \frac{F}{g M_0} = \frac{v~E \dot{M}}{g M_0}
\end{equation}
which allows us to write
\begin{equation}
  \label{eq:4}
  \frac{v(t)}{v~E} = \log( \frac{M_0}{M(t)} ) - \frac{1}{\psi} \qty( 1 - \frac{M}{M_0} )
\end{equation}

\section{Vertical range}
\label{sec:vertical-range}

To calculate the vertical range of a rocket we proceed as before by
integrating the velocity over $t$
\begin{align*} 
s &= \int v(t) \dd{t} = v~E \int_0^t \log( \frac{M_0}{M(t)} ) \dd{t} - \int_0^t g t \dd{t'} \\
&=v~E \frac{M_0}{\dot{M}} \qty[ 1 - \frac{M}{M_0} \qty( \log( \frac{M_0}{M(t)}) + 1 ) ] - \half g t^2 \\
&=v~E \frac{M_0}{\dot{M}} \qty[ 1 - \frac{M}{M_0} \qty( \log( \frac{M_0}{M(t)}) + 1 ) ] \\ & \quad- \frac{g}{2} \frac{M_0^2}{\dot{M}^2} \qty( 1 - \frac{M}{M_0})^2
\end{align*} assuming $g$ is constant.

\section{Launch aspects}
\label{sec:launch-aspects}

Consider the kinetic energy of the payload,
\begin{align*}
  \half M \Delta v^2 &= \half M v~E^2 \log^2\qty(\frac{M_0}{M}) \\ &= \half M v~E^2 \log^2\qty(R)
\end{align*}
For $R$ the rocket parameter. Normalising this by the rest-frame
energy of the propellant in the rocket,
\[ \half (M_0-M) v~E^2 \]
we find the ratio
\begin{equation}
\label{eq:5} \phi(R) = \frac{M}{M_0-M} \log^2(R) = \frac{\log^2(R)}{R-1} 
\end{equation}

\begin{figure}[b]
  \centering
  \begin{tikzpicture}
    \begin{axis}[xmin=0, ymin=0, height = 5cm, width=\columnwidth, xlabel=$R$, ylabel=$\phi(R)$]
      \addplot[mark=none, draw=muted-green, ultra thick, domain=1:10] {(ln(x)*ln(x))/(x-1)};
    \end{axis}
  \end{tikzpicture}
  \caption{$\phi(R)$ as a function of the rocket parameter, $R$.}
  \label{fig:rocket-ratio}
\end{figure}

By differentiating we find the optimal $R$ to be
\begin{equation}
  \label{eq:6}
  \dv{\phi}{R} = 0 = - \frac{\log^2(R)}{(R-1)^2} + \frac{2 \log(R)}{R(R-1)}
\end{equation}
So the optimal $R = R~C$ satisfies
\begin{equation}
  \label{eq:7}
  \log{R~C} = 2 - \frac{2}{R~C}
\end{equation}
so $R~C \approx 4.9$.

%%% Local Variables: 
%%% mode: latex
%%% TeX-master: "../project"
%%% End: 
