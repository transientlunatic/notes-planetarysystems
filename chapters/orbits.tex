Recall that a gravitational orbit can be defined by the expression
\begin{equation}
  \label{eq:1}
  \frac{1}{r} = \frac{GM}{h^2} (1+ e \cos(\theta))
\end{equation}
Where $M$ is the mass of the body being orbited, $r$ is the radial
coordinate, and $\theta$ the angular coordinate, $h = v_0 r_0$ is the
angular momentum per unit mass of the satellite, and eccentricity
\begin{equation}
  \label{eq:2}
  e = \frac{r_0 v_0^2}{GM} - 1
\end{equation}
which is clearly dependent upon the specific energy of the satellite. 

A circular orbit can be attained by having $e=0$, so \[ v_0 =
\sqrt{\frac{GM}{r_0}} \]
While a parabolic orbit occurs when $e=1$, so
\[ v_0 = \sqrt{\frac{2GM}{r_0}} \]

\section{Elliptical transer orbits}
\label{sec:ellipt-trans-orbits}

These orbits are used to move to a higher (or lower) orbit, by
producing an impulse parallel to the trajectory. An impulse is made in
the smaller orbit, at a point which becomes the perigee of an ellipse,
and a second one is made once the desired radius of the outer orbit
has been obtained at the apogee, and a new circular orbit is set up.

\begin{figure}[h]
  \centering
  \begin{tikzpicture}
    \draw [ultra thick] (0,0) circle (1.5);
    \draw [ultra thick] (0,0) circle (3);
    \draw [thick, muted-green] (0, 0.75) ellipse (2 and 2.25);
    \fill (0,0) circle (0.2);
    \draw [<->] (3.5,-1) -- (3.5,3) node [right, midway] {$a_1 + a_2$};
\draw [<->](0,0) -- (90:1.5) node [midway, right] {$a_1$};
\draw [<->] (0,0) -- (140:3) node [midway, above=0.3cm] {$a_2$};
  \end{tikzpicture}
  \caption{An elliptical transfer orbit}
  \label{fig:elliptical-transfer}
\end{figure}

The semi-major axis of the elliptical transfer orbit will be 
\[ a = \half \qty( a_1 + a_2) \] for $a_1$ and $a_2$ the radii of the
two circular orbits. The eccentricity is
\[ e = \frac{a_2 - a_1}{a_2 + a_1} = \frac{\alpha-1}{\alpha+1} \] for
$\alpha = a_2/a_1$, so
\[ 1 + e = \frac{2 \alpha}{\alpha+1} \]
The two circular orbits have velocities
\[ v_1 = \sqrt{\frac{GM}{a_1}}, \qquad v_2 = \sqrt{\frac{GM}{a_2}} \]
The required delta-vee at the perigee is then
\begin{align*}
  \Delta v_1 = v~{e,1} - v_1 &= \sqrt{\frac{GM}{a_1}} (1+e)^{\half} - \sqrt{\frac{GM}{a_1}} \\
&= \sqrt{\frac{GM}{a_1}} \qty[ (1+e)^{\half} -1 ]
\end{align*}
and the delta-vee at the apogee, to reinject into a circular orbit is then
\begin{align*}
  \Delta v_2 = v_2 - v~{2,e} &= \sqrt{\frac{GM}{a_2}} - \sqrt{\frac{GM}{a_2}} (1-e)^{\half} \\
&= \sqrt{\frac{GM}{a_2}} \qty[ 1 - (1-e)^{\half}]
\end{align*}
The total boost in terms of $v_1$ is then
\begin{align*}
  \frac{\Delta v}{v_1} = \frac{\Delta v_1 + \Delta v_2}{v_1} &=
    (1+e)^{\half} - 1 + \sqrt{\frac{a_1}{a_2}} \qty[1-(1-e)^{\half}] \\
 &= (1+e)^{\half} - 1 + \sqrt{\frac{1-e}{1+e}} \qty[ 1 + (1-e)^{\half}] \\
 &= \sqrt{\frac{2 \alpha}{1+\alpha}} - 1 + \frac{1}{\sqrt{\alpha}} \qty[ 1 - \sqrt{\frac{2}{1+\alpha}}]
\end{align*}
We can write the total energy per spacecraft mass as
\[ C = \half v^2 - \frac{\mu}{r} \] For $v^2 = \mu \qty(\frac{2}{r} -
\frac{1}{a})$, and $\mu=GM$.
The energy of the two orbits is then
\begin{align*}
  C_1 &= \half v_1^2 - \frac{\mu}{a_1} = - \half \frac{\mu}{a_1} \\
  C_2 &= - \half \frac{\mu}{a_2}
\end{align*}
Thus the change in energy is
\begin{equation}
  \label{eq:3}
  \Delta C = C_2 - C_1 = \frac{\mu}{2} \qty( \frac{1}{a_1} - \frac{1}{a_2})
\end{equation}
This is the minimum change in kinetic energy required to affect the transfer.

The transfer orbit will be
\[ C~T = - \frac{\mu}{2 a~T} = - \frac{\mu}{a_1+a_2} \]
The energy increments are then
\begin{align*}
  \Delta C~A &= C~T - C_1 = - \frac{\mu}{a_1+a_2} + \frac{\mu}{2 a_1} = \frac{\mu}{2 a_1} \frac{a_2-a_1}{a_2+a_2} \\
\Delta C~B &= C_2 - C~T = - \frac{\mu}{2 a_2} + \frac{\mu}{a_1+a_2} = \frac{\mu}{2a_2} \frac{a_2-a_1}{a_2+ a_1}
\end{align*}
The energy changes are kinetic, so
\begin{align*}
  \Delta C~A &= \half \qty( v~A + \Delta v~A)^2 - \half v~A^2 \\
\Delta C~B &= \half v~B^2 - \half \qty(v~B - \Delta v~B)^2
\end{align*}
Equating the expressions for $C~A$, 
\[ \frac{\mu}{2 a_1} \frac{a_2-a_1}{a_2+a_1} = \half \qty( v~A + \Delta v~A)^2 - \half v^2~A \]
Then
\begin{align}
  \Delta v~A &= \qty[ v~A^2 + \frac{\mu}{a_1} \frac{a_2-a_1}{a_2+a_1}]^{\half} - v~A \nonumber\\
&= v~A \qty( \qty[1+ \frac{\mu}{a_1 v~A^2} \frac{a_2-a_1}{a_2+a_1}]^{\half}-1) \nonumber\\
&= \sqrt{\frac{\mu}{a_1}} \qty[ \qty( \frac{2a_2}{a_2+a_1} )^{\half}-1]
\end{align}
similarly,
\begin{equation}
  \label{eq:4}
  \Delta v~B = \sqrt{\frac{\mu}{a_2}} \qty[ 1-\qty(\frac{2 a_1}{a_2+a_1})^{\half}-1]
\end{equation}
for $v~A = \sqrt{\mu/a_1}$.

The mass required to produce these boosts can then be found from the rocket equation,
\begin{align*}
  \Delta v~A &= v~E \log(\frac{M_0}{M~A}) \\
  \Delta v~B &= v~E \log(\frac{M~A}{M~B})
\end{align*}
The total boost required will then be
\begin{align*} \Delta v = \Delta v~A + \Delta v~B &= v~E \log( \frac{M_0}{M~A} ) + v~E \log( \frac{M~A}{M~B} ) \\ &= v~E \log( \frac{M_0}{M~B} ) 
\end{align*}

Finally, the time taken for the transfer can be found by calculating
how long it takes to travel along half of the ellipse. For this we
need Kepler's third law, 
\begin{equation}
  \label{eq:6}
  T^2 = \frac{4 \pi^2}{GM} a^3
\end{equation}
So the transfer time, $\tau$ is
\begin{equation}
  \label{eq:7}
  \tau = \half T = \pi \qty( \frac{a^3}{GM})^{\half}
\end{equation}
If the spacecraft is being transferred to a moving object, such as a
planet the mutual motion must be accounted for, so, if the planet has
a period T~P, then
\[ 2 \pi \frac{\tau}{T~P} = 2 \pi^2 \qty(\frac{a^3}{GM})^{\half} \frac{1}{2\pi} \qty( \frac{GM}{a_2^3})^{\half} = \pi \qty( \frac{a}{a_2})^{\frac{3}{2}} \]
and so the manoeuvre must being when the angle to the target is
\[ \pi - \pi \qty( \frac{a}{a_2})^{\frac{3}{2}} =\pi-\pi \qty(\frac{\tau}{T_2}) \]

%%% Local Variables: 
%%% mode: latex
%%% TeX-master: "../project"
%%% End: 
