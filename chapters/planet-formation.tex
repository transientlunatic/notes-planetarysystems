In our Solar System all of the planets orbit anticlockwise, with the
sun rotating in the same direction, and all of the orbital planes are
close to coplanar (except Mercury). Their orbits are very close to
circular (except Mercury, with $e=0.2$), and with the exception of
Venus and Uranus, they all rotate anticlockwise. Each planet is
approximately twice the distance from the sun of the previous one, and
most satellites orbit anticlockwise.

\section{The nebular hypothesis}
\label{sec:nebular-hypothesis}

In the nebular hypothesis the sun and planets formed from a cloud of
interstellar material, with the sun forming at the centre of a
flattened high density region. The planets then form out of the
remainder of the cloud. There are a number of questions arising from
this idea though. How does a lat solar system form, and how do the
planets form?

The Jeans criterion states that compressive perturbations of a gas
with uniform density, $\rho$, and sound speed $c~s$ over distances
greater than the Jeans length, $\Lambda$,
\begin{equation}
  \label{eq:35}
  \Lambda = \qty( \frac{\pi c~s^2}{G \rho})^{\half}
\end{equation}
where $G$ is the gravitational constant. For a one solar mass cloud
core with a temperature of $10\,\kelvin$ and a radius of
$0.1\,\text{pc}$ this occurs over a timescale of around one million
years

Radio measurements of gas clouds show that they do possess angular
momentum, and have rotational periods of around 20 million years. The
centrifugal force per unit mass of the gas is then $j^2/r^3$, for $r$
the distance from the axis, and so the elements are balanced within a
disk.

\section{Evolution of  a stellar nebula}
\label{sec:evol-astell-nebula}

Consider the condition for centrifugal balance,
\begin{equation}
  \label{eq:36}
  \frac{j^2}{r^3} = \pdv{\Phi}{r}
\end{equation}
where the gravitational force is 
\[ - \pdv{\Phi}{r} \]
so the energy of an element of gas, per unit mass, is
\begin{equation}
  \label{eq:37}
  E = \frac{j^2}{2r^2} + \Phi
\end{equation}
Thus the change of total energy due to interaction between two gas
elements is
\begin{equation}
  \label{eq:38}
  \dd{E} = \frac{j_1 \dd{j_1}}{r^2} - \frac{j_1^2 \dd{r_1}}{r_1^3} + \dd{\Phi_1} + \frac{j_2 \dd{j_2}}{r^2} - \frac{j_2^2 \dd{r_2}}{r_2^3} + \dd{\Phi_2}
\end{equation}
Considering the conservation of momentum, and the balance of forces,
\begin{equation}
  \label{eq:39}
  \dd{E} = \qty( \frac{j_1}{r^2} - \frac{j_2}{r^2} )\dd{j_1} = \qty( \Omega_1 - \Omega_2 ) \dd{j_1}
\end{equation}
Planets form through a multi-step process, with solid grins condensing
out of the gas of the nebula, which accrete into larger bodies:
planetesimals, which then collide and coalesce to form protoplanets.

Dust particles then start to settle into the plane of the
protoplanetary disk, with the gravitational force in the $z$-direction
being
\begin{equation}
  \label{eq:40}
  \frac{GM m~g z}{R^3}
\end{equation}
where $R$ is the distance of the grain from the star, and $m~g$ is its
mass. The angular velocity will then be
\begin{equation}
  \label{eq:41}
  \Omega^2 = \frac{GM}{R^3}
\end{equation}
so the acceleration on each grain is
\begin{equation}
  \label{eq:42}
  g_z = \Omega^2 z
\end{equation}

As each dust particle falls towards the centre of the disk it
experiences drag from the gas it travels through. For a spherical
particle of radius $a$ travelling at a velocity $v$ through a gas of
density $\rho$ this drag is
\begin{equation}
  \label{eq:43}
  F~{drag} = \pi a^2 \rho c~s v
\end{equation}
for $c~s$ the sound speed of the gas. The equation of motion for the
grain is then
\begin{align}
  m~g \dv{v}{t} &= F~{drag} - m~g g_z \\
\frac{4 \pi a^3 \rho~g}{3} \dv{v}{t} &= \pi a^2 \rho c~s v - \frac{4 \pi a^3 \rho~g}{3} \Omega^2 z
\end{align}
for grains of density $\rho~g$.

At some point the grain will reach a terminal velocity, when its
acceleration reaches $0$, so
\begin{equation}
  \label{eq:44}
  v = \frac{4 a \rho~g \Omega^2 H}{3 \rho c~s} \sim \frac{4 a \rho~g \Omega}{3 \rho} \sim \frac{8 a \rho~g \Omega H}{3 \Sigma}
\end{equation}
for $z=H$, the scale height of the disk from its mid-point, an
isothermal sound speed, $c~s \approx H \Omega$, and $\Sigma$ the
surface density, $\Sigma = 2 H \rho$. The settling time is then
\begin{equation}
  \label{eq:45}
  \tau~s \sim \frac{H}{v} \sim \frac{3 \Sigma}{8 a \rho~g \Omega} \sim \frac{\Sigma}{16 a \rho~g}\,\text{orbits}
\end{equation}
For $\Sigma \sim 10^3\,\kilogram\,\meter^{-2}$, $\rho~g \sim
10^3\,\kilogram\,\meter^{-3}$, $a = 10^{-5}\,\meter$, we get $10^4$
orbits, or $10^5$ years at a distance $5\,\text{AU}$.

\section{The formation of planetessimals}
\label{sec:form-plan}

Dust particles collide and coalesce in the midplane of the disk, and
in the time scale of the settling time they grow into objects with a
radius of between $1$ and $1000\,\kilo\meter$, at which point the
gravitational effect of the objects becomes significant. Each
planetessimal will sweep up material about a radius $r$, and so
accrete matter at a rate
\begin{equation}
  \label{eq:46}
  \dv{m~c}{t} = v_0 n m \pi r^2
\end{equation}
for particles of initial velocity $v_0$, number density $n$, and mass
$m$. By conservation of energy,
\begin{align*}
  \half m v_0^2 &= \half mv^2 - \frac{Gm~c m}{R~c} \\ 
v^2 &= v_0^2 + \frac{2 G m~c}{R~c} \\
v &= \sqrt{v_0^2 + \frac{2G m~c}{R~c}} \\
\dv{m~c}{t} &= v_0 n m \pi R~c^2 \qty( 1 + \frac{2 G m~c}{R~c v_0^2} )
\end{align*}

The timescale for the planetesimal to double its mass is
\begin{equation}
  \label{eq:47}
  \tau~{double} = m~c \qty( \dv{m~c}{t})^{-1}
\end{equation}
From the previous expression we can see when $Gm~c / R \approx v_0^2$,
this is proportional to $m~c^{1/3}$, and so smaller planetesimals grow
faster than large ones, and we have orderly growth. However, when
$Gm~c / R~c \gg v_0^2$ the gravitational force domiantes and $\tau~c
\propto m~c^{-1/3}$, and we have runaway growth.

\section{Isolation mass}
\label{sec:isolation-mass}

The next consideration is how large a planetary embryo can grow by
accreting other planetesimals. To find this, define an annulus of
half-width
\[ \Delta a~{max} = C R~{Hill} \] for $R~{Hill}$ the radius of the
Hill sphere where the embryo's gravitational influence dominates its
parent star, with
\begin{equation}
  \label{eq:48}
  R~{Hill} = a \qty( \frac{m~c}{3 M_{*}} )^{\frac{1}{3}}
\end{equation}
The isolation mass, where accretion stops, with $\Sigma~s$ the surface density of planetesimals, and $a$ the semimajor axis of the orbit, is
\begin{equation}
  \label{eq:49}
  m~{iso} = 2 \pi a \Sigma~s \Delta a~{max}
\end{equation}
Substituting in the half-width
\begin{align*}
  m~{iso} &= 2 \pi a \Sigma~s \Delta a~{max} \\
&= 2 \pi a \Sigma~s C a \qty( \frac{m~{iso}}{3 M_{*}} )^{1/3} \\
&= \sqrt{8 \pi^3} a^3 \Simga~s^{\frac{3}{2}} C^{\frac{3}{2}} \qty( 3 M_{*})^{- \frac{1}{2}}
\end{align*}
Them, taking $C=2 \sqrt{3}$, $\Sigma~s = 100\,\kilo\gram\,\meter^{-2}$
at $a = 1\,\text{AU}$, $m~{iso} = 0.07 M~E$. Thus terrestrial planets
did not form simply by embryos accreting matter, but this model is in
line with the estimated core masses of Saturn, Neptune, and
Uranus. For terrestrial planets the perturbations of the outer planets
are likely to have been significant.

\section{Formation of gas giants}
\label{sec:formation-gas-giants}

The core accretion model for gas giant formation is favoured, but a
second, gravitational instability model also exists.

\subsection{Core accretion}
\label{sec:core-accretion}

In the core accretion model a gas giant starts as the rock and ice
planetary embryo considered in the previous model, but this core
becomes so large that it is capable of supporting a gas envelope,
which remains at hydrostatic equilibrium until it reaches a critical
mass (around $15\,M~E$), after which it contracts on the
Kelvin-Helmholtz timescale (around $10^6$ years). After this a phase
of rapid gas accretion occurs until the planet is sufficiently massive
to open a gap in the protoplanetary disk. Radial planetary migration
can change the timescale of this process however.

The critical mass can be found from the relation
\begin{equation}
  \label{eq:50}
  \frac{M~{crit}}{M~E} \approx 12 \qty( \frac{\dot{M}~{core}}{10^{-6}\,M~e/\text{year}} )^{\frac{1}{4}} \qty( \frac{\kappa~R}{1\,\centi\meter\,\gram^{-1}} )^{\frac{1}{4}}
\end{equation}
for $\kappa~R$ the Rosseland mean opacity across all frequencies,
weighted by a Planckian distribution.

\subsection{Gravitational instability model}
\label{sec:grav-inst-model}

In this model the self-gravity of the gas in the protoplanetary disk
is sufficient to cause the disk to fragment and form clusters which
can then form gas giants. For this to work a high surface density is
required, which is most likely to occur early in the evolution of the
disk, and there are doubts as to whether there is sufficient time for
gas giants to form in this period. This model also favours the
formation of the planets at large radii (50 to 100\,AU).

For a rotating disk the Toomre $Q$-parameter quantifies the stability
of a system; for gravitational instability 
\[ Q \equiv \frac{c~s \Omega}{\pi G \Sigma} \lesssim 1 \] Taking the
speed of sound to be $c~s \simeq 0.5\,\kilo\meter/\second$ we would
require $\Sigma \approx 1.4\e{3}\,\gram\,\centi\meter^{-2}$. Assuming
a characteristic wavelength for a gravitational instability to be
\[ \lambda~{crit} = \frac{2 c~s^2}{G \Sigma} \]
then the mass of objects formed through disk fragmentation would be 
\[ M~p \sim \pi \Sigma \lambda~{crit}^2 \sim \frac{4 c~s^4}{G^2
  \Sigma} \sim 3\,M~J \] Approaching the point of fragmentation the
disk is likely to start forming spiral arms due to differential
rotation, allowing heat to transfer through the gas faster, increasing
the sound speed, and transferring angular momentum, leading to a lower
surface density. This the cooling time is an important
consideration. For the annulus of the disk the Kelvin-Helmholtz time
scale is
\[ t~{cool} = \frac{U}{2 \sigma T~{disk}^4} \] for $U$ the thermal
energy per unit surface area of the disk. If the cooling time is less
than the rotation period the heat can be radiated away and the disk is
likely to collapse.

\section{Planetary migration}
\label{sec:planetary-migration}

Planets will not necessarily remain in the orbits they formed in, and
orbits can change due to perturbations from larger planets,
interactions with planetesimals, and mutual interactions between
planets within the system.

As the newly formed planet interacts with the gas disk the gas can
develop vorticity, producing waves of higher gas density. These can
impart angular momentum on a planet if the have the same frequency as
the planet's gravitational potential, setting up a resonance. Density
waves from behind will exert a torque on the planet greater than the
torque from waves within the planet's orbit, forcing it into a lower
orbit. This is type I gas disk migration.

Type II gas disk migration occurs when high-mass planets interact
tidally with the gas disk as they accrete material. This tidal force
strongly perturbs the disk, and the exchange of angular momentum
between the planet and disk repels gas from the orbit, sweeping out an
annular gap in the gas disk. The planet will now move radially inward
as it accretes matter, and stops when it reaches the inner edge of the
gap. When a planet is large enough to open a gap orbital evolution
will occur; the radial velocity of the gas disk is
\[ v~r = -\frac{\dot{M}}{2 \pi r \Sigma} \] If there i a tidal barrier
on the outer edge of the gap the evolution of the disk will cause the
orbit to shrink provided that the local disk mass exceeds the planet's
mass. The rate of orbital shrinkage is equal to the radial velocity,
so the timescale is
\begin{equation}
  \label{eq:51}
  \tau_0 = \frac{2}{3 \alpha} \qty( \frac{h}{r} )^{-2} \Omega^{-1}
\end{equation}


%%% Local Variables: 
%%% mode: latex
%%% TeX-master: "../project"
%%% End: 
