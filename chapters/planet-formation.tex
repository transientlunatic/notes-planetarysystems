In our Solar System all of the planets orbit anticlockwise, with the
sun rotating in the same direction, and all of the orbital planes are
close to coplanar (except Mercury). Their orbits are very close to
circular (except Mercury, with $e=0.2$), and with the exception of
Venus and Uranus, they all rotate anticlockwise. Each planet is
approximately twice the distance from the sun of the previous one, and
most satellites orbit anticlockwise.

\section{The nebular hypothesis}
\label{sec:nebular-hypothesis}

In the nebular hypothesis the sun and planets formed from a cloud of
interstellar material, with the sun forming at the centre of a
flattened high density region. The planets then form out of the
remainder of the cloud. There are a number of questions arising from
this idea though. How does a lat solar system form, and how do the
planets form?

The Jeans criterion states that compressive perturbations of a gas
with uniform density, $\rho$, and sound speed $c~s$ over distances
greater than the Jeans length, $\Lambda$,
\begin{equation}
  \label{eq:35}
  \Lambda = \qty( \frac{\pi c~s^2}{G \rho})^{\half}
\end{equation}
where $G$ is the gravitational constant. For a one solar mass cloud
core with a temperature of $10\,\kelvin$ and a radius of
$0.1\,\text{pc}$ this occurs over a timescale of around one million
years

Radio measurements of gas clouds show that they do possess angular
momentum, and have rotational periods of around 20 million years. The
centrifugal force per unit mass of the gas is then $j^2/r^3$, for $r$
the distance from the axis, and so the elements are balanced within a
disk.

\section{Evolution of  astellar nebula}
\label{sec:evol-astell-nebula}

Consider the condition for centrifugal balance,
\begin{equation}
  \label{eq:36}
  \frac{j^2}{r^3} = \pdv{\Phi}{r}
\end{equation}
where the gravitational force is 
\[ - \pdv{\Phi}{r} \]
so the energy of an element of gas, per unit mass, is
\begin{equation}
  \label{eq:37}
  E = \frac{j^2}{2r^2} + \Phi
\end{equation}
Thus the change of total energy due to interaction between two gas
elements is
\begin{equation}
  \label{eq:38}
  \dd{E} = \frac{j_1 \dd{j_1}}{r^2} - \frac{j_1^2 \dd{r_1}}{r_1^3} + \dd{\Phi_1} + \frac{j_2 \dd{j_2}}{r^2} - \frac{j_2^2 \dd{r_2}}{r_2^3} + \dd{\Phi_2}
\end{equation}
Considering the conservation of momentum, and the balance of forces,
\begin{equation}
  \label{eq:39}
  \dd{E} = \qty( \frac{j_1}{r^2} - \frac{j_2}{r^2} )\dd{j_1} = \qty( \Omega_1 - \Omega_2 ) \dd{j_1}
\end{equation}
Planets form through a multi-step process, with solid grins condensing
out of the gas of the nebula, which accrete into larger bodies:
planetesimals, which then collide and coalesce to form protoplanets.

Dust particles then start to settle into the plane of the
protoplanetary disk, with the gravitational force in the $z$-direction
being
\begin{equation}
  \label{eq:40}
  \frac{GM m~g z}{R^3}
\end{equation}
where $R$ is the distance of the grain from the star, and $m~g$ is its
mass. The angular velocity will then be
\begin{equation}
  \label{eq:41}
  \Omega^2 = \frac{GM}{R^3}
\end{equation}
so the acceleration on each grain is
\begin{equation}
  \label{eq:42}
  g_z = \Omega^2 z
\end{equation}

As each dust particle falls towards the centre of the disk it
experiences drag from the gas it travels through. For a spherical
particle of radius $a$ travelling at a velocity $v$ through a gas of
density $\rho$ this drag is
\begin{equation}
  \label{eq:43}
  F~{drag} = \pi a^2 \rho c~s v
\end{equation}
for $c~s$ the sound speed of the gas. The equation of motion for the
grain is then
\begin{align}
  m~g \dv{v}{t} &= F~{drag} - m~g g_z \\
\frac{4 \pi a^3 \rho~g}{3} \dv{v}{t} &= \pi a^2 \rho c~s v - \frac{4 \pi a^3 \rho~g}{3} \Omega^2 z
\end{align}
for grains of density $\rho~g$.

At some point the grain will reach a terminal velocity, when its
acceleration reaches $0$, so
\begin{equation}
  \label{eq:44}
  v = \frac{4 a \rho~g \Omega^2 H}{3 \rho c~s} \sim \frac{4 a \rho~g \Omega}{3 \rho} \sim \frac{8 a \rho~g \Omega H}{3 \Sigma}
\end{equation}
for $z=H$, the scale height of the disk from its mid-point, an
isothermal sound speed, $c~s \approx H \Omega$, and $\Sigma$ the
surface density, $\Sigma = 2 H \rho$. The settling time is then
\begin{equation}
  \label{eq:45}
  \tau~s \sim \frac{H}{v} \sim \frac{3 \Sigma}{8 a \rho~g \Omega} \sim \frac{\Sigma}{16 a \rho~g}\,\text{orbits}
\end{equation}
For $\Sigma \sim 10^3\,\kilogram\,\meter^{-2}$, $\rho~g \sim
10^3\,\kilogram\,\meter^{-3}$, $a = 10^{-5}\,\meter$, we get $10^4$
orbits, or $10^5$ years at a distance $5\,\text{AU}$.

%%% Local Variables: 
%%% mode: latex
%%% TeX-master: "../project"
%%% End: 
