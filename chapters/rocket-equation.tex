
\section{Derivation of the rocket equation}
\label{sec:deriv-rock-equat}

In order to change its momentum a rocket ejects mass; the Tsiolkovsky
rocket equation relates the complications of the changing mass and the
rocket's dynamics.  To analyse the motion of the rocket we start at
Newton's second law,
\begin{equation}
  \label{eq:10}
  \sum_i F_i = \dv{p(t)}{t}
\end{equation}
In the case of a rocket both the mass and the momentum have a time
dependence, and
\begin{equation}
  \label{eq:11}
  \dv{p}{t} = \lim_{\Delta t \to 0} \qty( \frac{p_2 - p_1}{\Delta t} )
\end{equation}
Consider $p_1 = (m + \Delta m) V$, and $p_2 = m(V + \Delta v) + \Delta
m V~e$, where $V~e$ is the velocity of the exhausted mass.

In the frame of an observer,
\[ V~e = V - v~e \]
thus 
\begin{align*}
  p_2 - p_1 &= m(V + \Delta v) + \Delta m (V-v~e) - (m+\Delta m) V \\
&= m \Delta v - \Delta m \ v~e
\end{align*}
let $\dd{m} = -\Delta m$, so
\begin{align*}
  \dv{p}{t} &= \frac{m \Delta v + v~e \dd{m}}{\Delta t} \\
&= m \dv{v}{t} + v~e \dv{m}{t} = m \dot{v} + v~e \dot{m}
\end{align*}
Since there are no external forces, and assuming that $v~e$ is
constant,
\begin{align*}
  F &= m \dot{v} + v~e \dot{m} = 0 \\
m \dot{v} &= - v~e \dot{m} \\
\Delta v &= \int v~e m \dd{m} 
\end{align*}
which gives
\begin{fequation}[Tsiolkovsky rocket equation]
  \label{eq:13}
\label{eq:rocket-equation}
  \Delta v= v~e \log( \frac{m_0}{m(t)} )
\end{fequation}
Now define the mass fraction as the proportion of the propellant which
has been expended,
\begin{definition}[Mass fraction]
  \label{def:mass-fraction}
  \[ M~f = 1 - \qty[ \frac{m(t)}{m_0}] \]
\end{definition}

\section{Thrust}
\label{sec:thrust}

If we make the assumption that $\dot{m}$ is constant, then
\[ m(t) = m_0 - \dot{m} t \]
and the total time for driven flight can then be found,
\begin{equation}
  \label{eq:14}
  t = \frac{m_0 - m(t)}{\dot{m}} = \frac{m_0}{\dot{m}} \qty( 1 - \frac{m}{m_0}) = \frac{m_0}{\dot{m}} M~f
\end{equation}
hence the range of the rocket is
\begin{align*}
  s(t) &= \int_0^t v(t') \dd{t'}\\ &= v~e \int_0^t \log( \frac{m_0}{m_0 - \dot{m} t'} ) \dd{t'} \\ &= v~e \frac{m_0}{\dot{m}}
\qty[\qty(1- \frac{\dot{m} t}{m_0}) \log( 1 - \frac{\dot{m} t}{m_0}) + \frac{\dot{m}}{m_0} t]\\
& \text{since } m(t) = m = m_0 - \dot{m} t, 1- \dot{m} t / m_0 = m(t)/m\\
&= v~e \frac{m_0}{\dot{m}} \qty[ \frac{m}{m_0} \log( \frac{m}{m_0}) +  1 - \frac{m}{m_0}] \\ 
&= v~e \frac{m_0}{\dot{m}} \qty[ 1 - \frac{m}{m_0} \qty( \log(\frac{m}{m_0}) + 1 )]
\end{align*}

\section{Equations of motion}
\label{sec:equations-motion}

We now have a set of equations to describe the rocket's motion,
\begin{align}
  \label{eq:15}
  v(t) &= v~e \log( \frac{m_0}{m} ) \\
  \label{eq:16}
  t &= \frac{m_0}{\dot{m}} M~f \\
  \label{eq:17}
  s(t) &= v~e \frac{m_0}{\dot{m}} \qty[1 - \frac{m}{m_0} \qty(\log( \frac{m}{m_0} )+1)] + v_i \frac{m_0}{\dot{m}} M~f
\end{align}

It is useful to define a new quantity here also:
\begin{definition}[Specific Impulse]
\label{def:specific-impulse}
  \[ I~{sp} = \frac{v~e}{g} \]
\end{definition}
which is just the exhaust velocity normalised by the gravitational
acceleration.

\begin{example}[A simple rocket]
  Consider the third stage burn of a rocket, which has an initial
  velocity $v~i = 2 \e{3}\, \meter\, \second^{-1}$, and contains $4
  \e{3}\, \kilogram$ of fuel, and has an empty mass of $700\,
  \kilogram$. The vacuum exhaust speed is $v~e =
  2930\,\meter\,\second^{-1}$, with
  $\dot{m}=100\,\kilogram\,\second^{-1}$. \\ How far does the rocket
  travel during the burn?  The mass ratio is $\frac{4.7}{0.7} =
  6.71$, so \[ \Delta v = v~e \log(6.71) =
  55.79\,\meter\,\second^{-1} \] and the burn time is
\[ \Delta t = \frac{4.7}{0.1 (1 - \frac{0.7}{4.7} )} \approx 40\,\second \]
Thus \begin{align*} s &= 2930\,\meter\second^{-1} &&\times \frac{4700\,\kilogram}{100\,\kilogram\,\second^{-1}}\\ &&& \times \qty[1 - \frac{1}{6.71} \qty( \log(6.71) + 1 )] \\&= 158.51 \kilo\meter 
\end{align*}
\end{example}

\section{Multistaging}
\label{sec:multistaging}

The structural mass of the rocket is a major source of inefficiency
during the burn, and one way to reduce its impact is to jettison
stages of the rocket as the fuel they are carrying is expended. This
allows a greater speed to be achieved with the same quantity of
fuel. Modern rockets seldom use more than three stages due to the
engineering compexity, however.

From the rocket equation (\ref{eq:rocket-equation}), we have
\[ v(t) = v~E \log( \frac{M_0}{M} ) + v~i \] for $M_0$ the initial
mass of the rocket, and $M$ its final mass, while $v~i$ is its
velocity at the start of the burn. For a single stage rocket the
velocity boost will be
\[ v(t) = v~E \log(R_0) \]
where 
\[ R_0 = \frac{M~S + M~f + M~p}{M~s + M~p} \] for $M~S$, $M~f$, and
$M~p$ respectively the masses of the structure, fuel, and payload.

If the rocket is composed of two stages with equal mass,
\[ v(t) = v~E \log(R_1) + v~E \log(R_2) \] which is larger than the
total speed achievable using a single-stage rocket.

%%% Local Variables: 
%%% mode: latex
%%% TeX-master: "../project"
%%% End: 
