Modern formualtions of mechanics and relativity require an extensive
quantity of linear algebra, often using constructions such as tensors.

\section{Vectors and Covectors}
\label{sec:vectors-covectors}

\begin{definition}[Vector]
  A vector is an object which is an element 
      \marginpar{$v^\alpha$}
  of a vector space; in terms of geometry this is an entity with both a concept of magntiude and of direction.
\end{definition}

\begin{definition}[Covector]

  A covector is the dual object of a vector, 
    \marginpar{$v_\alpha$}
    and maps a vector space to a scalar field.
\end{definition}

It is possible to visualise a vector as an arrow with length and
direction. In this case its corresponding covector is a means of
assigning its length, for example a series of evenly-spaced planes
through which it passes, or the notches on a tape measure.\\
Conventional notation represents the components of a vector as a
superscript, e.g. for the vector $v$ the components are
$v^\alpha$. For a covector they are subscript, $v_\alpha$.

\section{Tensors}
\label{sec:tensors}

It is possible to use multiplication by a scalar to change the
magnitude of a vector, but what if we wish to change its direction?
This is where we require tensors, which are objects composed out of
vectors.

Let's return to the concept of a vector field. A vector field is a
mathematical construction where every point in space has a vector
(think of it as an arrow) attached to it. Suppose now we want to
attach some more complicated construction to every point? For example,
what if we want a measure of the distortion of space at every point,
by attaching an ellipsoid to every point? This is where the concept of
a tensor field grows from a vector field.

\begin{definition}[Tensor]
  A $(k, l)$ \emph{tensor} is the map
  \[ T : \underbracket{ \vs{V^{*}} \otimes \cdots \otimes \vs{V^{*}}
  }_{k \text{ copies}} \otimes \overbracket{ \vs{V} \otimes \cdots
    \otimes \vs{V} }^{l \text{ times}} \to \mathbb{R}\] which is
  linear in all of its arguments.
\end{definition}

\section{Tensor Fields}
\label{sec:tensor-fields}

The concept of a tensor field is a two-stage concept. First there is the concept of the vector bundle.

\begin{definition}[Real Vector Bundle]
  This consists of
  \begin{enumerate}
  \item topological spaces, $X$, the base space, and $E$, the total
    space.
  \item a continuous surjection $\pi : E \to X$, the bundle projection,
  \item for every $x \in X$ the structure of a finite-dimensional real vector space on the fibre $\pi^{-1}(\set{x})$.
  \end{enumerate}
  where the compatibility condition is: for every point in $X$ there
  is an open neighbourhood $U$, a natural number $k$, and a
  homeomorphism,
  \[ \phi: U \times \mathbb{R}^k \to \pi^{-1}(U) \] such that for all
  $x \in U$,
  \begin{itemize}
  \item $( \pi \circ \phi )(x,v) = x$ for all vectors $v$ in
    $\mathbb{R}^k$, and
  \item the map $v \to \phi(x,v)$ is an isomorphism between the vector
    spaces $\mathbb{R}^k$ and $\pi^{-1}(\set{x})$.
  \end{itemize}
\end{definition}

An example of a vector bundle is a M\"obius strip, which is a line
bundle over a one-sphere $S^1$ (i.e. a circle). Every local point in
the strip looks like $U \times \mathbb{R}$, for $U$ an open arc which
includes the point. However, the total bundle is different from $S^1
\times \mathbb{R}$, which is a cylinder. This makes it locally
trivial, but the global geometry is more complex.\\
The fibre bundle can thus be seen as a vector space which is dependent
on some parameter in a manifold, $M$; in the case of the M\"obius strip, this is the
angular distance around the strip.

\subsection{Manifolds}
\label{sec:manifolds}

A \emph{manifold} is a topological space which resembles Euclidean
space around every point on it. In one dimension examples include
lines and circles, which are homeomorphic to $\mathbb{E}^1$, but
figures-of-eight are a counter-example, due to the crossing of the
lines. In two dimensions planes, spheres, and tori are examples, while
the Klein bottle is a counter-example. The local geometry of manifolds
is Euclidean---consider a football field, where it is possible to draw
straight lines and have a flat surface despite it being located on the
surface of a sphere, but the global geometry often is not. It is
possible to map regions of a manifold onto the Euclidean plane,
producing \emph{charts} by means of \emph{map projections}. In a
region which appears on two neighbouring charts a transformation will
be required to move between the two, and this is a \emph{transition
  map}.

Consider a point, $p$ on the manifold $M$. The \emph{tangent space} is
a vector space at the point $p$ which is at a tangent to the manifold
$M$, and can be denoted $T_{x}(M)$ or $T_xM$. The tangent bundle is
the disjoint union of all of the tangent spaces across the manifold,
and is denoted $TM$.

Now, at any given point a tensor field assigns a tensor to the space,
\[ \vs{V} \otimes \cdots \otimes \vs{V} \otimes \vs{V^{*}} \otimes
\cdots \otimes \vs{V^{*}}\] where $\vs{V}$ is the tangent space, and
$\vs{V^{*}}$ the cotangent space at the point.

%%% Local Variables: 
%%% mode: latex
%%% TeX-master: "../project"
%%% End: 

