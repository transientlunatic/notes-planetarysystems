A planet is an object which orbits a star or a stellar remnant,
defined in 2006 by the IAU as celestial body which
\begin{itemize}
\item is in orbit around the Sun,
\item has sufficient mass for its self-gravity to overcome rigid body
  forces so that it assumes a hydrostatic equilibrium (nearly round)
  shape, and
\item has cleared the neighbourhood around its orbit.
\end{itemize}
an exoplanet is a planet in orbit about a star other than our own.

The earliest attempts to detect exoplanets came from Huygens in 1698,
while the first successful detection was made in 1992, with the
discovery of a planet orbiting a pulsar. The first main sequence star
found to have planets was 50 Persei in 1995.

There are a number of methods used to detect exoplanets:

\begin{description}
\item[Direct observation] Planets are identified by the reflected
  radiation, either through optical, infra-red, or radio emission, and
  its polarisation.
\item[Doppler Method] Radial velocity changes of the star with respect
  to the Earth can be deduced from the star's spectral lines.
\item[Astrometry] By measuring the position of a star in the sky the
  'wobble' in the star's relative motion caused by the planet can
  reveal the presence of planets.
\item[Transit Method] The effect of the planet on the observed
  luminosity of a star as it passes in front of the disk can be
  measured.
\item[Gravitational Microlensing] The gravitational field of a star
  acts like a lens, brightening the background stars.
\item[Pulsar Timing] The presence of a planet introduces a delay in
  the arrival of pulses.
\end{description}

\section{Direct Observation}
\label{sec:direct-observation}

\subsection{Albedo}
\label{sec:albedo}

The albedo is the ability of an object to reflect light. \\
Let the radiation flux density on the surface of the Sun be $F~\sun$,
and so the flux density, $F~p(r)$ at a distance $r$ from the star is
\begin{equation}
  \label{eq:1}
  F~p(r) = F~\sun \frac{R^2_{\sun}}{r^2}
\end{equation}
with $R_{\sun} /\ r$ being the angular diameter of the Sun at a
distance $r$. Then the total flux on the surface of the planet then becomes 
\[ L = \pi R^2~p F~p = \pi R^2~p F_{\sun} \frac{R^2_{\sun}}{r^2} \]

\subsection{Bond Albedo}
\label{sec:bond-albedo}

The Bond Albedo, $A$, is defined as the ratio of the emergent flux to
the incident flux. The flux reflected by the planet is

\begin{equation}
  \label{eq:2}
  L = L^{\prime} \cdot A =  \pi R^2~p A F_{\sun} \frac{R^2_{\sun}}{r^2}
\end{equation}

If a planet is a distance $\Delta$ from the observer its observed
flux, $F$, will be
\begin{equation}
  \label{eq:3}
  F = \frac{L^{\prime}}{4 \pi \Delta^2}
\end{equation}

However, reflection is anisotropic, so the flux must be corrected by a
factor $C \Phi(\alpha)$ which depends on the phase angle of the
planet. The function $\Phi$ is the phase function, and is normalised
such that $\Phi(\alpha = 0) = 1$. We also require the normalising
constant $C$ such that
\begin{equation}
  \label{eq:4}
  \frac{ C \int_S \Phi(\alpha) \dd{S}}{4 \pi \Delta^2} = 1
\end{equation}

With this, the true observed flux at a distance $\Delta$ will be
\begin{equation}
  \label{eq:5}
  F = \frac{C \Phi(\alpha) L^{\prime}}{4 \pi \Delta^2} 
    = \frac{C \Phi(\alpha)}{4 \pi \Delta^2} A \pi R^2~p F_{\sun} \frac{R_{\sun}^2}{r^2}
\end{equation}

Now, since
\begin{align*}
  \int_S \Phi(\alpha) \dd{S} & = \Delta^2 \int_{\alpha=0}^{\pi} \int_{\phi=0}^{2\pi} \Phi(\alpha) \sin(\alpha) \dd{\alpha} \dd{\phi} \\
                             & = 2 \Delta^2 \pi \int_0^{\pi} \Phi(\alpha) \sin(\alpha) \dd{\alpha}                                   \\
  \therefore C               & = \frac{2}{\int_0^{\pi} \Phi(\alpha) \sin(\alpha) \dd{\alpha}}
\end{align*}

Let $q = \frac{CA}{4}$. The Bond Albedo, $A$, can be expressed
\begin{equation}
  \label{eq:6}
  A = 2 p \int_{\alpha=0}^{\pi} \Phi(\alpha) \sin(\alpha) \dd{\alpha}
\end{equation}
with $p$ the geometric albedo, and $q$ is the phase integral:
\begin{equation}
  \label{eq:7}
  q = 2 \int_{\alpha=0}^{\pi} \Phi(\alpha) \sin(\alpha) \dd{\alpha}
\end{equation}

\subsection{Geometric Albedo}
\label{sec:geometric-albedo-1}

A Lambertian surface is a perfectly white diffuse surface which
reflects all incident radiation, with a phase function $\Phi(\alpha) =
\cos(\alpha)$. Thus
\begin{equation}
  \label{eq:8}
  C = \frac{2}{\int_0^{\pi} \Phi(\alpha) \sin(\alpha) \dd{\alpha}} 
    = \frac{2}{\int_0^{\pi/2} \cos(\alpha) \sin(\alpha)  \dd{\alpha} } = 4
\end{equation}
which implies that $A = p = 1$ for a Lambertian surface.

We can then use a Lambertian surface to measure the amount of light
reflected from another surface.  Take the observed flux density,
equation (\ref{eq:5}), and let $\alpha=0$. Now take the flux density
of the Lambertian surface,
\begin{equation}
  \label{eq:9}
  F~L(\alpha=0) = \frac{4}{4 \pi \Delta^2} \pi R~p^2 F_{\sun} \frac{R_{\sun}^2}{r^2} 
\end{equation}
since $A=1$. The ratio of these two gives $\frac{F}{F~L} = p$.
%%% Local Variables: 
%%% mode: latex
%%% TeX-master: "../project"
%%% End: 
