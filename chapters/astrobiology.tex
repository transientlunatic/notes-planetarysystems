In order to determine which planets, and what conditions, are likely
to support life we need a firm concept of the concept of life.

\section{Life}
\label{sec:life}

Terrestrial life is based on a number of organic molecules, composed
primarily from hydrogen, carbon, oxygen, and nitrogen. These include
fatty acids, carbohydrates, and amino acids. These elements are fairly
abundant in the universe: hydrogen exists as a result of the big bang,
oxygen is synthesised in core-collapse supernovae, and carbon and
nitrogen (and oxygen) are synthesised in the CNO process. As we know
it, life also requires a terrestrial planet with liquid water. This
requires a temperature in the range $270$ to $370\,\kelvin$, and a
sufficiently high atmospheric pressure to support it.

\section{Equilibrium temperature}
\label{sec:equil-temp}

The emission of stars and planets can be approximated as blackbody
radiation, so, the flux from a body at temperature $T$ is
\begin{equation}
  \label{eq:52}
  F = \sigma T^4
\end{equation}
for $\sigma$ the Stefan-Boltzmann constant, the total radiative power
is then
\begin{equation}
  \label{eq:53}
  P = 4 \pi R~p^2 \sigma T^4~p
\end{equation}
and a planet at a distance $r~p$ absorbs
\begin{equation}
  \label{eq:54}
  P = (1-A) \pi R~p^2 \qty( \frac{R_{\odot}}{r~p} )^2 \sigma T^4_{\odot}
\end{equation}
for an albedo, $A$, of the planet. Equating these,
\begin{equation}
  \label{eq:55}
  T~p = (1-A)^{\frac{1}{4}} \qty( \frac{R_{\odot}}{2 r~p} )^{\half} T_{\odot} \approx 279 (1-A)^{\frac{1}{4}} r~p^{-\half}
\end{equation}
So, for a general star of luminosity $L$,
\begin{equation}
  \label{eq:56}
  T~p = \qty( \frac{(1-A)L}{16 \pi \sigma r~p^2} )^{\frac{1}{4}}
\end{equation}

\section{The greenhouse effect}
\label{sec:greenhouse-effect}

The greenhouse effect can be incorporated by adding a term $\Delta T$
to equation (\ref{eq:56}). This is a result of the absorption of
infrared radiation which heats the atmosphere and in turn the surface
of the planet.

\section{Albedo}
\label{sec:albedo-1}

The terrain and atmosphere contribute strongly to the albedo of a
planet. For example for snow $A \approx 0.8$.

\section{Stability of the atmosphere}
\label{sec:stability-atmosphere}

A gas in thermodynamic equilibrium has a Maxwellian distribution of
velocities,
\[ F(v) \dd{v} \propto \exp( - \frac{m v^2}{2 kT} ) v^2 \dd{v} \]
and an average kinetic energy per particle
\[ \ev{T} = \frac{m}{2} \ev{v^2} = \frac{3 kT}{2} \]
so a root mean square velocity
\[ \ev{v^2}^{\half} = \qty(\frac{3kT}{m})^{\half} \] The escape
velocity of a planet is
\[ v~e = \sqrt{\frac{2GM}{R}} \]
so to retain an atmosphere in the long term 
\begin{equation}
  \label{eq:57}
  T \le \frac{GMm}{150 k R} 
\end{equation}
for a species of mass $m$, and assuming that the $v~e > 10 v~{rms}$.

%%% Local Variables: 
%%% mode: latex
%%% TeX-master: "../project"
%%% End: 
