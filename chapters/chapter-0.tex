%%%%%%%%%%%%%%%%%%%%%%%%%%%%
% CHAPTER 0                %
%%%%%%%%%%%%%%%%%%%%%%%%%%%%
% Special Relativity       %
%%%%%%%%%%%%%%%%%%%%%%%%%%%%

Special relativity is the extension of classical mechanics to
extremely high energy situations, and is based around just two axioms.

\begin{definition}[Event]
  An event is something which happens at a specific place at a
  particular instant in time.
\end{definition}

\begin{definition}[Inertial Reference Frame]
  An inertial reference frame is a means of assigning a position to an
  event. They have no acceleration, such that an inertial reference
  fram is a reference frame with respect to which Newton's Second Law
  holds.
\end{definition}

\subsection{The axioms of Special Relativity}
\label{sec:sraxiom}

\begin{enumerate}
\item All inertial reference frames are equivalent for the performance of all physical experiments.
\item The speed of light has the same, constant value in all reference frames.
\end{enumerate}
The first axiom is more commonly known as the \emph{principle of relativity}.

\subsection{Minkowski Diagrams}
\label{sec:minkdiag}

Any event can be described by four coordinates, $(t;x,y,z)$, but, by
choosing an appropriate reference frame we can describe this with just
two coordinates, $(t;x)$; a diagram of this situation is shown in
figure \ref{fig:minksimple}.

\begin{figure}
\centering
\begin{tikzpicture}[scale=1.5]
\fill [muted-blue] (.5, .5) circle (0.05);
\draw[help lines, ->] (-1,-1) -- (1.7,-1) node [below] {space, $x$};
\draw[help lines, ->] (-1, -1) -- (-1, 1.2) node [right] {time, $t$};
\end{tikzpicture}
\caption{An event on a Minkowski diagram.}
\label{fig:minksimple}
\end{figure}

\subsection{Invariant Interval}
\label{sec:invint}

In Euclidean space there is a concept of \emph{distance} which is
invariant; $x^2 + y^2 + z^2$ is the same under all transformations.
In spacetime there is a generalisation of this concept in the
\emph{invariant interval}:
\begin{equation}
  \label{eq:invariantinter}
  s^2 = \Delta t^2 - \Delta x^2
\end{equation}
The value of $s^2$ provides a simplified understanding of causality.
\begin{itemize}
\item $s^2 > 0$---time-like separation.
\item $s^2 = 0$---null separation (light-like).
\item $s^2 < 0$---space-like separation.
\end{itemize}

\begin{figure} \centering
\begin{tikzpicture}[scale=1.7]
\draw (-1,-1) -- (1,1) node [above] {$s^2 = 0$};
\fill [muted-orange, opacity=0.4] (-1,-1) -- (1,1) -- (-1,1) --cycle;
\draw (-.5,0) node {$s^2 > 0$};
\fill [muted-green, opacity=0.4] (-1,-1) -- (1,-1) -- (1,1) --cycle;
\draw (.5,0) node {$s^2 < 0$};
\draw[help lines, ->] (-1,-1) -- (1.7,-1) node [below] {space, $x$};
\draw[help lines, ->] (-1, -1) -- (-1, 1.2) node [right] {time, $t$};
\end{tikzpicture}
\caption{An event on a Minkowski diagram.}
\label{fig:minksimple}
\end{figure}

\subsection{Lorentz Transformations}
\label{sec:lorentz}

Consider two frames moving relative to one another, $S$ and
$S^\prime$. The frames are in standard configuration, i.e.\ an event
in $S$ at $(0,0)$ also occurs at $(0,0)$ in $S^\prime$. To convert
from an event occuring in $S$ to one in $S^\prime$ we use a Lorentz
transform. These have the form
\begin{subequations}
\begin{align}
  x^\prime &= x \cosh \phi - t \sinh \phi \\
t^\prime &= -x \sinh \phi + t \cosh \phi
\end{align}
\end{subequations}
and with 
\[ \phi(v) = \tanh[-1](v) \]
These can also be written
\begin{subequations}
  \begin{align}
    t^\prime &= \gamma (t-vx) \\
    x^\prime &= \gamma (x-vt)
  \end{align}
\end{subequations}
\begin{subequations}
  \begin{align}
    t &= \gamma (t^\prime + v x^\prime) \\
    x &= \gamma (x^\prime + v t^\prime)
  \end{align}
\end{subequations}

\subsection{Adding velocities}
\label{sec:addingvel}

Since $e^{\pm \phi} = \cosh(\phi) \pm \sinh(\phi)$,
\begin{subequations}
  \begin{align}
    t^\prime - x^\prime &= e^{\phi} (t-x) \\
    t^\prime + x^\prime &= e^{-\phi} (t+x)
  \end{align}
\end{subequations}

\subsection{Proper Time}
\label{sec:proper-time}

In the frame $S$ there are events at $(0,0)$ and $(t,x)$. A clock is
moving at a constant speed $v$, so that it is present at both
events. Pick $S^{\prime \prime}$, the clock's rest frame; $t^{\prime
  \prime}$ is thus the reading on the clock. This is denoted $\tau$,
the proper time. In any other frame in standard configuration we can then write
\begin{equation}
  \label{eq:1}
  \tau = \gamma(t-vx)
\end{equation}
The proper time will be agreed upon by all observers, and as such is a
Lorentz scalar. Since the velocity is $v = \frac{x}{t}$,
\[ t^2 - x^2 = \tau^2 = t^{\prime 2} - x^{\prime 2} \]

\subsection{Four-vectors}
\label{sec:four-vectors}

In $4$-dimensional spacetime the prototype displacement vector has the form
\[ \qty( \Delta t , \Delta x, \Delta y, \Delta z ) \]
and the transformation from one $4$-vector to another is 
\begin{equation}
  \label{eq:2}
  \begin{bmatrix}
    \Delta t \\ \Delta x \\ \Delta y \\ \Delta z
  \end{bmatrix}
=
\begin{bmatrix}
  \gamma & + \gamma v & 0 & 0 \\
+ \gamma v & \gamma &0 & 0 \\
0 & 0 & 1 & 0 \\
0 & 0 & 0 & 1
\end{bmatrix}
\begin{bmatrix}
  \Delta t^{\prime} \\ \Delta x^{\prime} \\ \Delta y^{\prime} \\ \Delta z^{\prime}
\end{bmatrix}
\end{equation}

\begin{equation}
  \label{eq:2}
\begin{bmatrix}
  \Delta t^{\prime} \\ \Delta x^{\prime} \\ \Delta y^{\prime} \\ \Delta z^{\prime}
\end{bmatrix}
=
\begin{bmatrix}
  \gamma & - \gamma v & 0 & 0 \\
- \gamma v & \gamma &0 & 0 \\
0 & 0 & 1 & 0 \\
0 & 0 & 0 & 1
\end{bmatrix}
  \begin{bmatrix}
    \Delta t \\ \Delta x \\ \Delta y \\ \Delta z
  \end{bmatrix}
\end{equation}

It is more normal to denote each component of the 4-vector as $x^n$,
$n \in \{0,1,2,3 \}$, or collectively, $x^{\mu}$.

\subsection{Four-vector inner products}
\label{sec:four-vector-inner}

The inner product of two four-vectors is somewhat less straightforward
that a three-vector scalar product.

\begin{definition}[Four-vector inner product]
  \begin{equation}
    \label{eq:3}
    \braket{A}{B} = \sum_{\mu, \nu} \eta_{\mu \nu} A^{\mu} B^{\nu}
  \end{equation}
  for $\eta_{\mu \nu} = \diag(1, -1, -1, -1)$ which is the metric
  tensor for a non-accelerating reference frame.
\end{definition}

\subsection{Velocity and Acceleration Vectors}
\label{sec:veloc-accel-vect}

The displacement 4-vector, $\Delta x^{\mu}$, transforms properly, as
does the infinitessimal displacement, $\dd{x^{\mu}}$. The proper time,
$\tau$, is the Lorentz scalar, so we can divide each component of
$\dd{x^{\mu}}$ by $\dd{\tau}$ and get a new vector, the velocity,
\[ U = \qty( \dv{x^0}{\tau}, \dv{x^1}{\tau}, \dv{x^2}{\tau}, \dv{x^3}{\tau} ) \]
and likewise, the acceleration,
\[ A = \qty( \dv[2]{x^0}{\tau}, \dv[2]{x^1}{\tau}, \dv[2]{x^2}{\tau}, \dv[2]{x^3}{\tau} ) \]

Now, since $t = \gamma \tau$,
\[ U^{\mu} = \qty( \gamma c , \gamma \vec{v} ) \]


\subsection{Momentum}
\label{sec:momentum}

Given the four-velocity we can find a four-momentum as well.
\[ p^{\mu} = (\gamma m c, \gamma m \vec{v} ) = \qty(\frac{E}{c}, \vec{p} )\]
The norm of any four-vector will be Lorentz invariant, so 
\[ p^2 = \frac{E^2}{c^2} - \vec{p}^2 \]
In the particle's rest frame,
\[ p^{\mu} = (mc, \vec{0}) \therefore p^2 = m^2 c^2 \]
and since it is Lorentz invariant, it must be the case in all frames.
Thus
\[ \frac{E^2}{c^2} - \vec{p}^2 = m^2 c^2  \quad \therefore \quad E^2 - \vec{p}^2 c^2 = m^2 c^4 \]

\subsection{Invariant Mass}
\label{sec:invariant-mass}

In scattering interactions the \emph{invariant mass} is a useful
quantity. Given a pair of particles with momenta $p_1$ and $p_2$, the
squared invariant mass, when working in natural units, is
\begin{align}
m_{12}^2  & = (p_1 + p_2)^2 \nonumber                                                                      \\
         & = (E_1+E_2, \vec{p}_1 + \vec{p}_2)^2 \nonumber                                                 \\
         & = (E_1+E_2)^2 - (\vec{p}_1 + \vec{p}_2) \cdot (\vec{p}_1 + \vec{p}_2) \nonumber                \\
         & = E_1^2 - \vec{p}_1^2 + E_2^2 - \vec{p}_2^2 + 2E_1 E_2 - 2 \vec{p}_1 \cdot \vec{p}_2 \nonumber \\
         & = m_1^2 + m_2^2 + 2(E_1 E_2 - \vec{p}_1 \cdot \vec{p}_2) 
\end{align}

\begin{example}[Two particles, one at rest]
  $\vec{p}_2 = \vec{0}$, and $E_2 = m_2$. Thus
\[ m_{12}^2 = m_1^2 + m_2^2 + 2 m_2 E_1 \]
\end{example}
\begin{example}[The Centre of Momentum Frame]
  The total 3-mometum in this frame is $\vec{0}$, so,
  \[ \vec{p}^{\text{com}}_1 = - \vec{p}^{\text{com}}_2 \]
then
\[ m_{12}^2 = \qty( E_1^{\text{com}} + E_2^{\text{com}} )^2 \]
\end{example}
Thus, the invariant mass is the total energy in the centre of mass
frame.

\subsection{Particle Collisions}
\label{sec:particle-collisions}

The vast majority of information about fundamental particles comes
from collision experiments. There are two types of collider used in physics,
\begin{description}
\item[Fixed Target] The target is at rest, while high-energy particles are accelerated and fired into it. Examples include the Bevatron, and the SLAC
\item[Colliding Beam] Two accelerated beams are focussed and fired
  into one another, so that the particles collide, usually in the
  centre of mass frame. The LHC is an example.
\end{description}
In all particle interactions the 4-momentum is conserved, so
\begin{align*} 
p^{\mu}_1 + p^{\mu}_2 &= p^{\mu}_3 + p^{\mu}_4 \\
(p_1 + p_2)^2 &= (p_3 + p_4)^2
\end{align*}
During an interaction the original particles are annihilated, and then
new particles are created from the resulting energy. As a result, for
a final state to be achieved there must be enough energy available to
create the appropriate quantity of mass. Defining the Lorentz invariant, 
\[ s = (\vec{p}_1 + \vec{p}_2)^2 \] For a desired pair of particles to
be produced with masses $m_3$ and $m_4$ we then require
\[ \sqrt{s} \ge m_3 + m_4 \]

\begin{example}[Creating two particles]
  We wish to create two particles, each with mass $2 \giga
  \electronvolt$ by colliding two protons (each $1 \giga
  \electronvolt$).\\
  The minimum energy in the case of a fixed-target experiment must
  come entirely from the proton in the beam, so
  \[ E_1 = \frac{m_{12}^2 - m_1^2 - m_2^2}{2 m_2} = \frac{(m_3+m_4)^2
    - m_1^2 - m_2^2}{2 m_2} \] thus, $E_1 = 7 \giga \electronvolt$.
  In the case of colliding beams both particles collide in the
  centre-of-mass frame, and so
  \[ E = \frac{m_1+m_2}{2} \ge \frac{m_3+m_4}{2} = 2 \giga
  \electronvolt \]
\end{example}
Clearly a much higher beam energy is required in the case of a fixed
target to get the same results, but the engineering challenges
involved with colliding two beams have, in the past, made fixed-target
experiments more feasible.


